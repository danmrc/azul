\documentclass[]{article}
\usepackage{lmodern}
\usepackage{amssymb,amsmath}
\usepackage{ifxetex,ifluatex}
\usepackage{fixltx2e} % provides \textsubscript
\ifnum 0\ifxetex 1\fi\ifluatex 1\fi=0 % if pdftex
  \usepackage[T1]{fontenc}
  \usepackage[utf8]{inputenc}
\else % if luatex or xelatex
  \ifxetex
    \usepackage{mathspec}
  \else
    \usepackage{fontspec}
  \fi
  \defaultfontfeatures{Ligatures=TeX,Scale=MatchLowercase}
\fi
% use upquote if available, for straight quotes in verbatim environments
\IfFileExists{upquote.sty}{\usepackage{upquote}}{}
% use microtype if available
\IfFileExists{microtype.sty}{%
\usepackage{microtype}
\UseMicrotypeSet[protrusion]{basicmath} % disable protrusion for tt fonts
}{}
\usepackage[margin=1in]{geometry}
\usepackage{hyperref}
\hypersetup{unicode=true,
            pdftitle={Verificando Computacionalmente o Teorema de Perron-Frobenius Parte I: Três Provas para o Ponto Fixo de Brouwer},
            pdfauthor={Pedro Cavalcante},
            pdfborder={0 0 0},
            breaklinks=true}
\urlstyle{same}  % don't use monospace font for urls
\usepackage{graphicx,grffile}
\makeatletter
\def\maxwidth{\ifdim\Gin@nat@width>\linewidth\linewidth\else\Gin@nat@width\fi}
\def\maxheight{\ifdim\Gin@nat@height>\textheight\textheight\else\Gin@nat@height\fi}
\makeatother
% Scale images if necessary, so that they will not overflow the page
% margins by default, and it is still possible to overwrite the defaults
% using explicit options in \includegraphics[width, height, ...]{}
\setkeys{Gin}{width=\maxwidth,height=\maxheight,keepaspectratio}
\IfFileExists{parskip.sty}{%
\usepackage{parskip}
}{% else
\setlength{\parindent}{0pt}
\setlength{\parskip}{6pt plus 2pt minus 1pt}
}
\setlength{\emergencystretch}{3em}  % prevent overfull lines
\providecommand{\tightlist}{%
  \setlength{\itemsep}{0pt}\setlength{\parskip}{0pt}}
\setcounter{secnumdepth}{0}
% Redefines (sub)paragraphs to behave more like sections
\ifx\paragraph\undefined\else
\let\oldparagraph\paragraph
\renewcommand{\paragraph}[1]{\oldparagraph{#1}\mbox{}}
\fi
\ifx\subparagraph\undefined\else
\let\oldsubparagraph\subparagraph
\renewcommand{\subparagraph}[1]{\oldsubparagraph{#1}\mbox{}}
\fi

%%% Use protect on footnotes to avoid problems with footnotes in titles
\let\rmarkdownfootnote\footnote%
\def\footnote{\protect\rmarkdownfootnote}

%%% Change title format to be more compact
\usepackage{titling}

% Create subtitle command for use in maketitle
\providecommand{\subtitle}[1]{
  \posttitle{
    \begin{center}\large#1\end{center}
    }
}

\setlength{\droptitle}{-2em}

  \title{Verificando Computacionalmente o Teorema de Perron-Frobenius Parte I:
Três Provas para o Ponto Fixo de Brouwer}
    \pretitle{\vspace{\droptitle}\centering\huge}
  \posttitle{\par}
    \author{Pedro Cavalcante}
    \preauthor{\centering\large\emph}
  \postauthor{\par}
      \predate{\centering\large\emph}
  \postdate{\par}
    \date{2019-07-21}


\begin{document}
\maketitle

\section{Pontos Fixos}\label{pontos-fixos}

Eu sou fascinado por um ramo de matemática que podemos chamar de
\emph{Teoria de Ponto Fixo}. É uma teoria surpreendenemte rica, com
aplicações vastas e em áreas muito diferentes, de Geometria Algébrica e
Ciência da Computação à Economia. As aplicações em economia são
particularmente interessantes para quem estuda Equilíbrio Geral ou
Teoria dos Jogos. Interessados podem conferir Border (1989).

Como eu quero que este post seja auto-contido, vou tentar introduzir os
conceitos necessários aqui. Leitores já familiarizados podem pular para
a próxima seção.

\begin{itemize}
\tightlist
\item
  \emph{Definição:} Diz-se que uma função \(f: A \to B\) admite um
  \emph{ponto fixo} se existe \(x \in A\) tal que \(f(x) = x\). \(x\) é
  dito \emph{ponto fixo} de \(f\). Notamos o conjunto de pontos fixos de
  \(f\) por \(\mathbb{F} (f)\).
\end{itemize}

No plano cartesiano, nosso velho amigo, uma função admite ponto fixo se
cruza o gráfico da \emph{função identidade \(g(x) = x\)}. A
cardinalidade do conjunto de pontos fixos (ou simplesmente o número de
pontos fixos) de uma função qualquer nesse contexto é o número de vezes
em que a identidade e esta função se cruzam. A função identidade
\(g(x) = x\) tem infinitos pontos fixos, algumas funções como
\(f(x) = 5 - x\) têm apenas um ponto fixo e outras como \(f(x) = x + 2\)
não têm pontos fixos.

Verificar a (in)existência de pontos fixos em funções afins é bem
simples, mas essas funções são raramente interessantes. Procurando por
padrões na existência ou não de pontos fixos nasceram vários teoremas.
Eu já falei de um deles aqui no blog, o
\href{https://azul.netlify.com/2018/10/31/banach/}{Teorema do Ponto Fixo
de Banach} que nos garante que se uma função é uma contração e tem
domínio em um espaço métrico completo, então tem \emph{um} e somente um
ponto fixo. A caixinha de ferramentas de teoremas de ponto fixo é ampla,
mas por enquanto vamos focar em um resultado:

\begin{itemize}
\tightlist
\item
  \emph{Teorema (Brouwer):} Seja \(A\) um subconjunto fechado, limitado
  e convexo do \(\mathbb{R}^n\) e \(f: A \to B\) uma função contínua.
  Então \(f\) admite pelo menos um ponto fixo.
\end{itemize}

O Teorema do Ponto Fixo de Brouwer é um dos fatos fundamentais da
topologia. O enunciado é curiosamente simples para uma conclusão tão
forte - embora possa não parecer a princípio. Imagine um disco que
contenha a sua borda. Deforme ele como quiser desde que não faça buracos
nem entradas. Deforme um disco até que vire um quadrado inscrito no
disco original, vire-o ao contrário, crie uma versão menor dele, o que
sua mente pensar. Se você não criar buracos nem entradas (portanto
deformando-o continuamente e sem ferir a convexidade do conjunto),
sempre haverá \emph{pelo menos um ponto} que fica parado.

Existem dezenas de maneiras conhecidas de provar este teorema. Algumas
provas extremamente simples podem ser construídas se aceitarmos como
verdadeiros alguns resultados como o Lema da Não-Retração. Outras muito
curiosas podem usar o fato de que o jogo Hex não tem empates, como a de
Gale (1979). Como provar o caso geral pode fugir ao escopo de um post de
blog, vou apresentar três demonstrações para casos particulares que
ilustram facetas diferentes e interessantes do problema. O leitor
interessado em demonstrações exaustivas deste resultado pode conferir
Buxton (2016).

\subsection{\texorpdfstring{O Teorema ``Óbvio'' de Ponto
Fixo}{O Teorema Óbvio de Ponto Fixo}}\label{o-teorema-obvio-de-ponto-fixo}

Este teorema é ``óbvio'' porque pode ser demonstrado até mesmo com
desenhos em um papel. Sua demonstração formal depende apenas de um fato
muito simples do cálculo, o Teorema do Valor Intermediário, e daí
decorre em uma sequência muito simples de encadeamentos lógicos. Neste
contexto, demonstrar a proposição é equivalente a mostrar que nenhuma
linha que vá de um lado de um quadrado até o oposto sem dar pulos
necessariamente cruza a diagonal.

\begin{figure}
\centering
\includegraphics{http://i.imgur.com/Mfe1XJo.png}
\caption{}
\end{figure}

O teorema que ajuda a amarrar a demonstração é razoavelmente simples
também.

\begin{itemize}
\tightlist
\item
  \emph{Teorema (Valor Intermediário):} Seja \(f\) uma função contínua,
  \(a\) e \(b\) números reais tal que \(a < b\). Então para todo
  \(d \in (f(a),f(b))\) existe \(c \in (a,b)\) tal que \(f(c) = d\).
\end{itemize}

Este teorema tem uma interpretação visual bem simples:

\begin{figure}
\centering
\includegraphics{https://upload.wikimedia.org/wikipedia/commons/thumb/e/e3/Teorema_do_valor_intermediario.svg/1024px-Teorema_do_valor_intermediario.svg.png}
\caption{}
\end{figure}

Agora podemos prosseguir para a demonstração em si.

\begin{itemize}
\item
  \emph{Proposição (Brouwer no Fechado \([0,1]\)):} Seja \(I=[0,1]\) um
  intervalo unitário fechado. Suponha \(f\) um mapeamento contínuo de
  \(I\) em \(I\). Então para pelo menos um \(x \in I\) vale que
  \(f(x)=x\).
\item
  \emph{Demonstração:} Poderíamos ter dois casos ``triviais'', em que
  \(f(0)=0\) ou \(f(1)=1\). Se não tivermos tais casos, então temos um
  mapeamento que associa todos os pontos de \(I\) a um subconjunto de
  \(I\) tal que \(f(0)>0\) e/ou \(f(1) < 1\) . Se definirmos um
  mapeamento \(g(x)= f(x) - x\), temos que \(g(0) > 0\) e \(g(1) < 0\).
  Segue do Teorema do Valor Intermediário que existe algum \(p \in I\)
  tal que \(g(p) = 0 \iff f(p) - p = 0\). Se \(f(p) - p = 0\) então
  \(f(p)=p\).
\end{itemize}

\subsection{Provando com o Lema da
Não-Retração}\label{provando-com-o-lema-da-nao-retracao}

\subsection*{Uma demonstração com
Combinatória}\label{uma-demonstracao-com-combinatoria}
\addcontentsline{toc}{subsection}{Uma demonstração com Combinatória}

\hypertarget{refs}{}
\hypertarget{ref-border}{}
Border, Kim C. 1989. \emph{Fixed Point Theorems with Applications to
Economics and Game Theory}. Cambridge University Press.

\hypertarget{ref-buxton}{}
Buxton, Colin. 2016. ``Brouwer Fixed-Point Theorem.''

\hypertarget{ref-gale}{}
Gale, David. 1979. ``The Game of Hex and the Brouwer Fixed-Point
Theorem.'' \emph{The American Mathematical Monthly} 86 (10). Taylor \&
Francis: 818--27.


\end{document}
